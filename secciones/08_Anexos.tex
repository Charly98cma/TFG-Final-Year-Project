\renewcommand{\thesection}{\Alph{section}}

\chapter{Anexo} \label{chp:anexo}

A continuación se muestra el código, las figuras y las métricas de interés procedentes del
desarrollo de este proyecto\footnote{Se puede acceder a las trazas de las pruebas, 
código de los programas desarrollados, y cualquier recurso utilizado en este TFG bajo
petición al autor.}.

%%%%%%%%%%%%%%%%%%%%%%%%%%%%%%%%%%%%%%%%%%%%%%%%%%%%%%%%%%%%%%%%%%%%%%%%%%%%%%%%
%%%%%%%%%%%%%%%%%%%%%%%%%%%%%%%%%%%%%%%%%%%%%%%%%%%%%%%%%%%%%%%%%%%%%%%%%%%%%%%%

\section{Datos en crudo} \label{sct:anexo_datosencrudo}

\subsection*{Primeras 100 líneas de la traza resultante del generador}

\begin{lstlisting}[style=consola]
[
	{
		"eventType" : "S",
		"id" : "rmi://10.0.1.1:1099/Broker1-M23",
		"contextFunction" : {},
		"constraints" : {
			"class:str" : [{"=" : "broadcast"}]
		}
	},
	{
		"eventType" : "S",
		"id" : "rmi://10.0.1.1:1099/Broker1-M24",
		"contextFunction" : {},
		"constraints" : {
			"class:str" : [{"=" : "id7471312"}]
		}
	},
	{
		"eventType" : "P",
		"class:str" : "broadcast"
	},
	{
		"eventType" : "P",
		"class:str" : "broadcast"
	},
	{
		"eventType" : "P",
		"class:str" : "broadcast"
	},
	{
		"eventType" : "P",
		"class:str" : "broadcast"
	},
	{
		"eventType" : "P",
		"class:str" : "broadcast"
	},
	{
		"eventType" : "S",
		"id" : "rmi://10.0.1.1:1099/Broker1-M36",
		"contextFunction" : {},
		"constraints" : {
			"class:str" : [{"=" : "__announce__"}]
		}
	},
	{
		"eventType" : "P",
		"class:str" : "__announce__"
	},
	{
		"eventType" : "P",
		"class:str" : "id11296229"
	},
	{
		"eventType" : "P",
		"class:str" : "ProxyID_-3884500000",
		"x:str" : "0.0",
		"y:str" : "0.0"
	},
	{
		"eventType" : "P",
		"class:str" : "id11296229"
	},
	{
		"eventType" : "P",
		"class:str" : "id11296229"
	},
	{
		"eventType" : "P",
		"class:str" : "id11296229"
	},
	{
		"eventType" : "P",
		"class:str" : "__announce__"
	},
	{
		"eventType" : "P",
		"class:str" : "id11296229"
	},
	{
		"eventType" : "P",
		"class:str" : "id11296229"
	},
	{
		"eventType" : "P",
		"class:str" : "id11296229"
	},
	{
		"eventType" : "P",
		"class:str" : "id11296229"
	},
	{
		"eventType" : "P",
		"class:str" : "id11296229"
	},
	{
		"eventType" : "P",
		"class:str" : "id11296229"
	},
	{
\end{lstlisting}

\subsection*{Primeras 100 líneas de la traza resultante que conforma la \autoref{fig:subsworkload_th_fast_spike_25k-113k-100k}} \label{ssct:anexo_fast-spike}

\begin{lstlisting}[style=consola]
"timestamp","throughput"
30.414,100401
30.912,100401
31.419,98619
31.917,100401
32.414,100603
32.913,100200
33.412,100200
33.919,98619
34.412,101419
34.917,99009
35.473,89928
35.913,113636
36.413,100000
36.917,99206
37.417,100000
37.918,99800
38.417,100200
38.914,100603
39.416,99601
39.917,99800
40.416,100200
40.913,100603
41.417,99206
41.916,100200
42.416,100000
42.916,100000
43.416,100000
43.916,100000
44.415,100200
44.916,99800
45.412,100806
45.916,99206
46.416,100000
46.912,100806
47.436,95419
47.912,105042
48.416,99206
48.915,100200
49.416,99800
49.916,100000
50.416,100000
50.914,100401
51.416,99601
51.938,95785
52.415,104821
52.918,99403
53.416,100401
53.916,100000
54.417,99800
54.916,100200
55.415,100200
55.913,100401
56.412,100200
56.915,99403
57.415,100000
57.913,100401
58.417,99206
58.912,101010
59.414,99601
59.917,99403
60.416,100200
60.916,100000
61.415,100200
61.916,99800
62.415,100200
62.916,99800
63.416,100000
63.913,100603
64.417,99206
64.915,100401
65.416,99800
65.915,100200
66.416,99800
66.916,100000
67.417,99800
67.916,100200
68.416,100000
68.916,100000
69.416,100000
69.915,100200
70.416,99800
70.916,100000
71.416,100000
71.912,100806
72.416,99206
72.916,100000
73.412,100806
73.944,93984
74.417,105708
74.916,100200
75.414,100401
75.916,99601
76.416,100000
76.916,100000
77.416,100000
77.915,100200
78.416,99800
78.915,100200
79.412,100603
\end{lstlisting}


%%%%%%%%%%%%%%%%%%%%%%%%%%%%%%%%%%%%%%%%%%%%%%%%%%%%%%%%%%%%%%%%%%%%%%%%%%%%%%%%
%%%%%%%%%%%%%%%%%%%%%%%%%%%%%%%%%%%%%%%%%%%%%%%%%%%%%%%%%%%%%%%%%%%%%%%%%%%%%%%%

\section{Datos adicionales} \label{sct:anexo_datosadicionales}

\subsection*{Scripts en R}

\begin{lstlisting}[language=R, caption={Programa en R que genera una carga de subscripciones incremental, como en la \autoref{fig:subs_workload-growing}}]
library(ggplot2)
library(dplyr)
library(tikzDevice)

# Function to export plot to LaTeX
print_LaTeX_plot <- function(path, plot, width, height) {
  tikz(filename = path, standAlone = TRUE, width = width, height = height)
  print(plot)
  dev.off()
  tools::texi2pdf(path, quiet = TRUE)
}

export_plots = TRUE
setwd(dirname(rstudioapi::getActiveDocumentContext()$path))

OUTPUT_CSV_DIR = "path/for/csv"
OUTPUT_PLOT_DIR = "/paht/for/plot/pdf"
low_workload  = 50000 # Stable system
high_workload = 75000 # System saturates


amount_noise = 0

total_duration = 60
windows_per_second = 10
total_windows = total_duration * windows_per_second

chunk_size = total_windows / 10

# First chunk (low steady)
first_chunk <- data.frame(subscriptions = rep(low_workload, chunk_size))

# Second chunk (increase)
step = (high_workload - low_workload) / (chunk_size * 2)
second_chunk <- data.frame(subscriptions = seq(low_workload, high_workload, step))

# Last chunk (high steady)
last_chunk <- data.frame(subscriptions = rep(high_workload, chunk_size * 7))

# Combine all chunks
workload <- rbind(first_chunk, second_chunk, last_chunk)

# Add noise
x <- workload$subscriptions
corrupt <- rbinom(length(x), 1, 1)    # choose an average of 10% to corrupt at random
corrupt <- as.logical(corrupt)
noise <- rnorm(sum(corrupt), 0, amount_noise) # generate the noise to add
x[corrupt] <- x[corrupt] + noise      # about 10% of x has been corrupted
workload$subscriptions <- x

plot <- ggplot(workload, aes(x = c(1:nrow(workload)))) +
  geom_line(aes(y = subscriptions), size=1) +
  xlab("Tiempo (segundos)") +
  ylab("Subscripciones") +
  theme(legend.position = "bottom",
        legend.text = element_text(size = 18, face="bold"),
        legend.title = element_text(size = 18, face="bold"),
        axis.text.x = element_text(size = 18, face="bold"),
        axis.text.y = element_text(size = 18, face="bold"),
        axis.title.x = element_text(size = 18, face="bold"),
        axis.title.y = element_text(size = 18, face="bold"))
plot

print_LaTeX_plot(OUTPUT_PLOT_DIR, plot = plot, width = 10, height = 5)

# Export
workload <- as.integer(workload$subscriptions)
write.table(workload, OUTPUT_CSV_DIR, col.names = FALSE, row.names = FALSE, sep = ";")
\end{lstlisting}

\begin{lstlisting}[language=R, caption={Programa en R para obtener la carga representada en \autoref{fig:subs_workload-growing}}]
library(ggplot2)
library(tikzDevice)

# Function to export plot to LaTeX
print_LaTeX_plot <- function(path, plot, width, height) {
  tikz(filename = path, standAlone = TRUE, width = width, height = height)
  print(plot)
  dev.off()
  tools::texi2pdf(path, quiet = TRUE)
}
export_plots = TRUE
setwd(dirname(rstudioapi::getActiveDocumentContext()$path))

##### Datos del plot #####

FILE_PATH = "../../logs/reports"
FILE_NAME = "logs-subs-10reps.csv"
XLABEL = "Subscripciones"
YLABEL = "Throughput (notificaciones/s)"
THEME <- theme(legend.position = "bottom",
               legend.text  = element_text(size = 18, face="bold"),
               legend.title = element_text(size = 18, face="bold"),
               axis.text.x  = element_text(size = 18, face="bold"),
               axis.text.y  = element_text(size = 18, face="bold"),
               axis.title.x = element_text(size = 18, face="bold"),
               axis.title.y = element_text(size = 18, face="bold"))

##### Codigo generador del plot #####

data <- read.csv(paste(FILE_PATH, FILE_NAME, sep="/"), header = TRUE, sep = ",")

plot <- ggplot(data, aes(x = subscriptions)) +
  geom_line(aes(y = throughput), size=1) +
  xlab(XLABEL) +
  ylab(YLABEL) +
  scale_y_log10(breaks = c(1, 30000, 40000, 50000, 60000, 70000, 80000),
                     labels = c("1", "30k", "40k", "50k", "60k", "70k", "80k")) +
  scale_x_log10(breaks = c(1, 10,100, 1000, 100000),
                labels = c("1", "10", "100", "1k", "100k")) +
  

print_LaTeX_plot("./throughput_logsubs.tex", plot = plot, width = 10, height = 5)
\end{lstlisting}

\begin{lstlisting}[language=R, caption={Programa en R que genera una carga de subscripciones con un pico puntual, como en la \autoref{fig:subsworkload_20k-100k}}]
library(ggplot2)
library(dplyr)
library(tikzDevice)

# Function to export plot to LaTeX
print_LaTeX_plot <- function(path, plot, width, height) {
  tikz(filename = path, standAlone = TRUE, width = width, height = height)
  print(plot)
  dev.off()
  tools::texi2pdf(path, quiet = TRUE)
}

export_plots = TRUE
setwd(dirname(rstudioapi::getActiveDocumentContext()$path))


# SCRIPT PARAMETERS
OUTPUT_CSV_DIR = "path/for/csv"
OUTPUT_PLOT_DIR = "path/for/plot/pdf"
low_workload  = 25000  # Stable system
high_workload = 113904  # System saturates
fast_inc = TRUE
plain_spike = FALSE


amount_noise = 0

total_duration = 60
windows_per_second = 10
total_windows = total_duration * windows_per_second

chunk_size = total_windows / 10

# First chunk (low steady)
first_chunk <- data.frame(subscriptions = rep(low_workload, chunk_size*3))

# Second chunk (fast or slow increase)
step_up = (high_workload - low_workload) / ((chunk_size)*(fast_inc) ? 10 : 1)
second_chunk <- data.frame(subscriptions = seq(low_workload, high_workload, step_up))

if (plain_spike) {
  # Extra chunk (top steady)
  extra_chunk <- data.frame(subscriptions = rep(high_workload, chunk_size*0.5))
}

# Third chunk (decrease)
third_chunk <- data.frame(subscriptions = seq(high_workload, low_workload, -step_up))

# Last chunk (low steady)
last_chunk <- data.frame(subscriptions = rep(low_workload, chunk_size*3))

# Combine all chunks
if (plain_spike) {
  workload <- rbind(first_chunk, second_chunk, third_chunk, last_chunk)
}else {
  workload <- rbind(first_chunk, second_chunk, extra_chunk, third_chunk, last_chunk)
}

# Add noise
x <- workload$subscriptions
corrupt <- rbinom(length(x), 1, 1)    # choose an average of 10% to corrupt at random
corrupt <- as.logical(corrupt)
noise <- rnorm(sum(corrupt), 0, amount_noise) # generate the noise to add
x[corrupt] <- x[corrupt] + noise      # about 10% of x has been corrupted
workload$subscriptions <- x

plot <- ggplot(workload, aes(x = c(1:nrow(workload)))) +
  geom_line(aes(y = subscriptions), size=1) +
  xlab("Tiempo (segundos)") +
  ylab("Subscripciones") +
  theme(legend.position = "bottom",
        legend.text = element_text(size = 18, face="bold"),
        legend.title = element_text(size = 18, face="bold"),
        axis.text.x = element_text(size = 18, face="bold"),
        axis.text.y = element_text(size = 18, face="bold"),
        axis.title.x = element_text(size = 18, face="bold"),
        axis.title.y = element_text(size = 18, face="bold"))

plot

print_LaTeX_plot(OUTPUT_PLOT_DIR, plot = plot, width = 10, height = 5)

# Export
workload <- as.integer(workload$subscriptions)
write.table(
  workload,
  OUTPUT_CSV_DIR,
  col.names = FALSE,
  row.names = FALSE,
  sep = ";")
\end{lstlisting}

\begin{lstlisting}[language=R, caption={Programa en R para generar las gráficas del Throughput y del Tiempo de Respuesta.}]
library(ggplot2)
library(tikzDevice)

setwd(dirname(rstudioapi::getActiveDocumentContext()$path))


# Function to export plot to LaTeX
print_LaTeX_plot <- function(path, plot, width, height) {
  tikz(filename = path, standAlone = TRUE, width = width, height = height)
  print(plot)
  dev.off()
  tools::texi2pdf(path, quiet = TRUE)
}
export_plots = TRUE

##### Vars. and flag #####

# Paths of the dir with measurements and of each csv
DIR_PATH = "path/to/dir/of/measurements"
FILE_NAME_TH = "path/to/th.csv"
FILE_NAME_RT = "path/to/rt.csv"

# Specify plots
PLOT_TH = TRUE
PLOT_RT = TRUE

# Number of points on the plot
N_POINTS = 50

THEME <- theme(legend.position = "bottom",
               legend.text = element_text(size = 18, face="bold"),
               legend.title = element_text(size = 18, face="bold"),
               axis.text.x = element_text(size = 18, face="bold"),
               axis.text.y = element_text(size = 18, face="bold"),
               axis.title.x = element_text(size = 18, face="bold"),
               axis.title.y = element_text(size = 18, face="bold"))

##### THROUGHPUT #####

if (PLOT_TH) {

  # Name of plot labels and generated PDF
  OUT_FILE_TH = paste("./th_",FILE_NAME_TH,".tex", sep="")
  XLABEL_TH = "Tiempo (segundos)"
  YLABEL_TH = "Throughput (notifs/s)"

  data_th <- read.csv(paste(DIR_PATH, FILE_NAME_TH, sep="/"), header = TRUE, sep = ",")

  # Delete first values if needed (system not entirely up, so times are very high)
  data_th <- data_th[-c(0:3),]

  # Spit data frame in N_POINTS equal-sized data frames and find mean
  data_th$points <- cut(c(1:nrow(data_th)),
                        breaks=(nrow(data_th)/N_POINTS)*(0:N_POINTS),
                        labels=c(1:N_POINTS))

  # Mean for each point
  new_data_th <- aggregate(cbind(timestamp, throughput) ~ points,
                        data = data_th, FUN = mean, na.rm = TRUE)

  # To plot all points, replace 'new_data_th' with 'data_th'
  plot_th <- ggplot(data_th, aes(x = timestamp)) +
    geom_line(aes(y = throughput), size=1) +
    xlab(XLABEL_TH) +
    ylab(YLABEL_TH) +
    scale_y_log10() +
    THEME

  print_LaTeX_plot(OUT_FILE_TH, plot = plot_th, width = 10, height = 5)
}

##### RESPONSE TIME #####

if (PLOT_RT) {

  # Name of plot labels and generated PDF
  OUT_FILE_RT = paste("./rt_",FILE_NAME_RT,".tex", sep="")
  XLABEL_RT = "ID Notificacion"
  YLABEL_RT = "Tiempo de respuesta (segundos)"

  data_rt <- read.csv(paste(DIR_PATH, FILE_NAME_RT, sep="/"), header = TRUE, sep = ",")

  # Delete first values if needed (system not entirely up, so times are very high)
  data_rt <- data_rt[-c(0:1),]

  # Spit data frame in N_POINTS equal-sized data frames and find mean
  data_rt$points <- cut(c(1:nrow(data_rt)),
                        breaks=(nrow(data_rt)/N_POINTS)*(0:N_POINTS),
                        labels=c(1:N_POINTS))

  # Mean for each point
  new_data_rt <- aggregate(cbind(id, resp_time) ~ points,
                        data = data_rt,
                        FUN = mean,
                        na.rm = TRUE)

  plot_rt <- ggplot(data_rt, aes(x = id)) +
    geom_line(aes(y = resp_time), size=1) +
    xlab(XLABEL_RT) +
    ylab(YLABEL_RT) +
    THEME

  print_LaTeX_plot(OUT_FILE_RT, plot = plot_rt, width = 10, height = 5)
}

\end{lstlisting}

%%%%%%%%%%%%%%%%%%%%%%%%%%%%%%%%%%%%%%%%%%%%%%%%%%%%%%%%%%%%%%%%%%%%%%%%%%%%%%%%
%%%%%%%%%%%%%%%%%%%%%%%%%%%%%%%%%%%%%%%%%%%%%%%%%%%%%%%%%%%%%%%%%%%%%%%%%%%%%%%%
