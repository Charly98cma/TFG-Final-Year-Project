\chapter{Resultados y conclusiones} \label{chp:resultados}

En este capítulo final, se presentarán los resultados de este trabajo, 
las conclusiones personales del autor y el trabajo futuro del proyecto.

%%%%%%%%%%%%%%%%%%%%%%%%%%%%%%%%%%%%%%%%%%%%%%%%%%%%%%%%%%%%%%%%%%%%%%%%%%%%%%%%
%%%%%%%%%%%%%%%%%%%%%%%%%%%%%%%%%%%%%%%%%%%%%%%%%%%%%%%%%%%%%%%%%%%%%%%%%%%%%%%%

\section{Resultados} \label{sct:resultados_resultados}

Los resultados obtenidos de este trabajo muestran que, mediante la aplicación 
de la carga de trabajo real obtenida y la aplicación de los modelos
predictivos, se pueden obtener mejoras en el proceso de auto-escalado.

De igual forma, se ha implementado y probado de forma satisfactoria el generador
de cargas de trabajo real, lo que permite que si en un futuro se dispone de 
alguna otra carga real que pudiese ser de utilidad. Esto se consigue mediante 
el cambio en la configuración a la sintaxis correspondiente de la nueva carga 
de trabajo.

Como se ha mencionado en la \autoref{sct:modelos_modelos-pred}, al no haber 
podido aplicar los modelos de predicción, no se han podido obtener los 
resultados esperados, pero los conocimientos adquiridos durante la realización
de este trabajo, permitirán aplicar estos modelos predictivos para aplicar las
mejoras necesarias al sistema.

En general, los resultados han sido positivos, a pesar de no haber tenido tiempo
suficiente para poder llevar a cabo todas las tareas que se habían planteado en
un principio, debido principalmente a retrasos generados por problemas durante
el desarrollo y el análisis de los datos del sistema.


%%%%%%%%%%%%%%%%%%%%%%%%%%%%%%%%%%%%%%%%%%%%%%%%%%%%%%%%%%%%%%%%%%%%%%%%%%%%%%%%
%%%%%%%%%%%%%%%%%%%%%%%%%%%%%%%%%%%%%%%%%%%%%%%%%%%%%%%%%%%%%%%%%%%%%%%%%%%%%%%%

\section{Conclusiones personales} \label{sct:resultados_conclusiones}

El desarrollo de este proyecto ha sido muy gratificante, tanto personalmente como
académicamente, y me ha permitido aprender y desarrollarme como estudiante, desde
leer y entender documentos científicos hasta desarrollar código de forma eficiente
y profesional, poniendo en práctica tanto conocimientos obtenidos durante el grado
como aquellos obtenidos durante el desarrollo de este TFG.

Personalmente, me ha gustado mucho poder conocer y participar en el desarrollo de
este proyecto tan ambicioso y llamativo, así como conocer el estado actual de las
tecnologías relacionadas, tanto directa como indirectamente. El futuro de este
proyecto y los posibles resultados que se pueden obtener me resultan muy interesantes
y emocionantes.


%%%%%%%%%%%%%%%%%%%%%%%%%%%%%%%%%%%%%%%%%%%%%%%%%%%%%%%%%%%%%%%%%%%%%%%%%%%%%%%%
%%%%%%%%%%%%%%%%%%%%%%%%%%%%%%%%%%%%%%%%%%%%%%%%%%%%%%%%%%%%%%%%%%%%%%%%%%%%%%%%

\section{Trabajo futuro} \label{sct:resultados_trabajofuturo}

% Aplicar más modelos de pred.
El trabajo futuro de este proyecto, principalmente, abarca la aplicación de los 
modelos de predicción, en concreto, el de una carga incremental (se incrementa 
la carga de permanente), una carga puntual (incremento puntual de la carga), y
el de una carga no predecible (la carga se genera de forma aleatoria, 
en base a ciertos parámetros). Tras aplicar estos modelos, se analizarán los
resultados obtenidos y se compararán con los resultados de los actuales modelos
estadísticos de predicción para ver si se producen mejores predicciones con
estos modelos predictivos.

% Implementación de las mejoras
Siguiendo la aplicación de los modelos predictivos, las mejoras que se 
identifiquen a partir de los resultados de estos serán otro de los trabajos a
realizar en un futuro.

% Dependencias y versiones
De forma adicional, sería recomendable revisar y actualizar los paquetes y 
dependencias del proyecto, de forma que haya estabilidad y uniformidad en las 
versiones usadas, y se disponga de las últimas actualizaciones y mejoras de las
herramientas usadas, además de prevenir posibles vulnerabilidades, como la 
recientemente vulnerabilidad de Log4J\cite{web:log4j}.

% Unificar proyectos y docu
Finalmente, otra de las tareas futuras es la de unificar el proyecto E-SilboPS 
en uno solo, al estar actualmente compuesto por dos proyectos, integrando ambos
en uno solo y resolviendo cualquier problema de compatibilidad o conflicto, de
forma que este sea más manejable y pueda documentarse de forma extensa en el
futuro.