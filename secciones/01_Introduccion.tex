\chapter{Introducción} \label{chp:intro}

En este capítulo se presentan y explican la motivación, el contexto, los objetivos
específicos del trabajo realizado y la estructura de este documento, con la 
intención de proporcionar una visión general del mismo.

%%%%%%%%%%%%%%%%%%%%%%%%%%%%%%%%%%%%%%%%%%%%%%%%%%%%%%%%%%%%%%%%%%%%%%%%%%%%%%%%
%%%%%%%%%%%%%%%%%%%%%%%%%%%%%%%%%%%%%%%%%%%%%%%%%%%%%%%%%%%%%%%%%%%%%%%%%%%%%%%%

\section{Motivación del proyecto} \label{sct:intro_motivacion}

% - Digitalización
% - Paradigmas como el IoT, Smart Cities, ...
% - Sistemas dirigidos por eventos que dan soporte a estos paradigmas
% - Sistemas, como los pub/sub, se apoyan en el cloud computing para...
% - Cloud computing ofrece la ventaja de la elasticidad (explicar)
% - Elasticidad y al necesidad de la misma para nuestro tipo de sistema

En los últimos años se ha producido un incremento significativo en la 
digitalización de numerosos aspectos de la vida cotidiana de las personas como 
consecuencia de la popularización de nuevos paradigmas tecnológicos como las 
Smart Cities o el Internet of Things\footnote{Dispositivos físicos con sensores y 
software que conectan e intercambian datos con otros dispositivos y sistemas a 
través de Internet} (o IoT, por sus siglas).

Estos paradigmas se basan en la infraestructura proporcionada por los sistemas 
dirigidos por eventos, los cuales son de vital importancia para la transmisión
de datos e información en tiempo real, como ocurre con el IoT, ya que 
proporcionan una infraestructura y arquitectura mediante 
Cloud Computing\footnote{Servicios de computación a través de Internet} 
capaz de proveer de los recursos necesarios a sistemas distribuidos que mueven 
grandes cantidades de datos (por ejemplo, sistemas publicador/subscriptor o 
sistemas distribuidos de procesamiento de datos).

La ventaja del Cloud Computing para los sistemas publicador/subscriptor 
es la elasticidad que proporciona a dicho sistema, es decir, la capacidad de 
adapta los recursos  dedicados a este en base a sus necesidades de cómputo. Esta 
característica permite optimizar la utilización de los recursos de computación,
ya los entornos de Cloud Computing usualmente, se basan en el pago bajo 
demanda (\textit{pay-as-you-go}) de recursos dedicados.

Para aprovechar esta elasticidad, los sistemas desplegados en una infraestructura 
cloud, en concreto, los sistemas publicador/subscriptor, deben de implementar 
dicha elasticidad, para poder adaptar los recursos disponibles a la cantidad 
de trabajo entrante, al encontrarse en un entorno muy cambiante, donde
puede haber  subidas y caídas repentinas de trabajo, llevando al sistema a una 
desabastecimiento o sobre-abastecimiento de recursos, respectivamente.

Esta elasticidad (o auto-escalado) se consigue mediante la implementación de un 
auto-escalador (auto-scaler) en el sistema publicador/subscriptor, que 
llevará a cabo las operaciones de escalado de forma automática y óptima, mediante
la monitorización del uso de los recursos de computación, y para así saber cuánto ha 
de escalar (en número de recursos añadidos/quitados).
Existen diferentes paradigmas de auto-escalado, que llevan a cabo las operaciones 
de escalado de diferentes formas.

El sistema usado y sobre el que se ha estado trabajando en este proyecto es 
E-SilboPS\cite{tfm:victor2017}\cite{thesis:tesisVictor}\cite{thesis:tesisSVavassori}, 
un sistema publicador/subscriptor basado en contenido con cierto grado de 
elasticidad\footnote{Sistema que adapta sus recursos a la cantidad 
de trabajo entrante de forma automática.}. Este sistema es la continuación natural,
mediante la implementación de la elasticidad y de numerosas mejoras, de 
SilboPS\cite{thesis:tesisSVavassori}, un sistema publicador/subscriptor
basado en contenido, e inspirado por SIENA\cite{paper:siena}.

El principal problema con el desarrollo de este sistema es la dificultad de 
probar su comportamiento en una situación real mediante el uso de una carga de 
trabajo con datos reales, ya que la publicación de unos datos de este estilo 
conllevan problemas de seguridad, tanto para la propia empresa como para
los usuarios, ya que se expone la información sobre sus intereses, gustos, etc.

Con una carga de trabajo real, se puede analizar el comportamiento del sistema 
en un entorno real, con el objetivo de encontrar los puntos a mejorar, para 
aumentar la precisión de las predicciones y el escalado, siendo este el objetivo
de este trabajo y proyecto.

El trabajo desarrollado en este TFG es la continuación natural de previos TFMs y tésis 
doctorales\cite{tfm:victor2017}\cite{thesis:tesisVictor}\cite{thesis:tesisSVavassori}, 
los cuales desarrollaron y mejoraron el sistema E-SilboPS, usado en este TFG.

%%%%%%%%%%%%%%%%%%%%%%%%%%%%%%%%%%%%%%%%%%%%%%%%%%%%%%%%%%%%%%%%%%%%%%%%%%%%%%%%
%%%%%%%%%%%%%%%%%%%%%%%%%%%%%%%%%%%%%%%%%%%%%%%%%%%%%%%%%%%%%%%%%%%%%%%%%%%%%%%%

\section{Contexto del proyecto} \label{sct:intro_contexto}

\subsection{Sistemas publicador/subscriptor auto-escalables} \label{ssct:intro_contexto_sistpubsub}

Los sistemas publicador/subscriptor operan como mediadores entre las entidades 
que actúan de publicadores, que generan información que puede ser de 
interés para un usuario; y los propios usuarios, que especifican sus intereses
mediante las subscripciones enviadas a este sistema, por lo que solo 
reciben la información que encaja en dichos
intereses)\cite{paper:themanyfacesofpubsub}\cite{paper:e-streamhub}.
Este proceso se lleva a cabo comparando cada una de las subscripciones que el 
sistema tiene almacenadas, y generando una lista de los usuarios a los que hay
que enviar dicha publicación. 

Existen dos tipos de paradigmas dentro de los sistemas publicador/subscriptor, 
los basados en tema (\textbf{topic-based}), que permiten al usuario recibir 
todas las publicaciones relacionadas con el tema especificado; y basados en 
contenido (\textbf{content-based}), que proporcionan al usuario más medidas 
para establecer qué quiere recibir por medio de especificar pares clave-valor 
que se comprobarán contra el contenido de la publicación.

Estos sistemas pueden ser auto-escalables, es decir, adaptan sus recursos y 
dedicación de los mismos a la cantidad de trabajo que están recibiendo. 
Este auto-escalado puede ser de dos tipos, \textbf{reactivo}, por el cual el 
sistema escala\footnote{Aumento/Disminución de los recursos} cuando se detecta
un elevado/bajo consumo de cierto  recurso computacional, o \textbf{predictivo}, 
anticipando el aumento/disminución de recursos a la saturación/desuso de los 
mismos, maximizando la eficiencia del sistema y minimizando el malgasto.

\subsection{E-SilboPS} \label{ssct:intro_motivacion_esilbops}

Para desarrollar este proyecto, se ha utilizado, con el fin de poder poner en 
práctica las mejoras aplicables a los sistemas publicador/subscriptor, el 
sistema E-SilboPS, que es la continuación del trabajo de previas 
tesis doctorales \cite{thesis:tesisVictor}\cite{thesis:tesisSVavassori},
llevadas a cabo en el grupo de investigación \textit{CETTICO} \cite{web:cettico},
en la Escuela Técnica Superior de Ingenieros 
Informáticos\footnote{\href{https://www.fi.upm.es/}{https://www.fi.upm.es/}} 
de la Universidad Politécnica de 
Madrid\footnote{\href{https://www.upm.es/}{https://www.upm.es/}}.

El grupo de investigación \textit{CETTICO} desarrolla y 
lleva proyectos en diferentes líneas de investigación, proporcionando 
soluciones tecnológicas a diferentes problemas en sistemas distribuidos, 
minería de datos y tecnologías web, entre otros muchos; tanto para empresas 
y sus departamentos de I+D+i como para proyectos de investigación, como este.

Este Trabajo de Fin de Grado se centra en la mejora de un sistema 
distribuido publish/subscribe auto-escalable basado en contenido, mediante el
análisis de sus principales métricas de rendimiento y la aplicación de modelos
predictivos, con el objetivo último de mejorar la predicción de los recursos 
que necesita, tanto en cantidad como en tipo, de forma que la eficiencia del 
sistema sea la máxima en cada momento.


%%%%%%%%%%%%%%%%%%%%%%%%%%%%%%%%%%%%%%%%%%%%%%%%%%%%%%%%%%%%%%%%%%%%%%%%%%%%%%%%
%%%%%%%%%%%%%%%%%%%%%%%%%%%%%%%%%%%%%%%%%%%%%%%%%%%%%%%%%%%%%%%%%%%%%%%%%%%%%%%%

\section{Objetivos} \label{sct:intro_objetivos}

El principal objetivo de este trabajo es la mejora de un sistema de 
auto-escalado desarrollado para sistemas distribuidos, en concreto, para 
sistemas de publicador/subscriptor, mediante la implementación de modelos 
predictivos aplicados a cargas de trabajo reales.

De este objetivo general, se pueden extraer los siguientes más específicos, que
reflejan de forma más concisa el desarrollo del proyecto:

\begin{itemize}

    \item[•] Realizar un análisis de la situación actual de los sistemas publicador/subscriptor 
    auto-escalables, con el objetivo de conocer su funcionamiento y sus limitaciones, extrayendo 
    los datos más relevantes, desarrollando una visión detallada y adquiriendo un conocimiento
    general de estos sistemas.
    
    \item[•] Realizar un análisis de las cargas de trabajo actuales y en uso en los sistemas 
    auto-escalables ya desarrollados, extrayendo la información más relevante sobre ellos para
    determinar su posible uso en futuras pruebas.
    
    \item[•] Realizar un estudio de las cargas de trabajo actualmente en uso en sistemas 
    distribuidos, y buscar una carga de trabajo real, que se haya obtenido por medio de un 
    sistema real, y que sea compatible y exportable al sistema que se utilizará.
    
    \item[•] Realizar un análisis estático de la carga de trabajo escogida, extrayendo los datos más 
    relevantes, y comprobando su estructura y completa compatibilidad con el sistema que se utilizará,
    con el objetivo de traducir dicha carga a una carga compatible con el input del sistema a 
    utilizar, por medio de desarrollar un traductor que produzca una carga de trabajo basada en la 
    escogida, y sea compatible con la entrada del sistema a utilizar.
    
    \item[•] Implementar pruebas que utilicen la carga de trabajo real escogida, con el objetivo de
    obtener las medidas de rendimiento relevantes del sistema, para analizar su comportamiento y 
    obtener conclusiones de su eficiencia.

    \item[•] Crear nuevas cargas de trabajo reales, partiendo de los resultados de las pruebas y
    de la carga de trabajo ya presente, de forma que dichas cargas lleven al sistema a diferentes 
    situaciones cuyo estudio, y el sus medidas de rendimiento más relevantes, es de interés para el 
    trabajo actual.

    \item[•] Desarrollar e implementar modelos predictivos que se adapten a las diferentes
    situaciones a las que el sistema se puede enfrentar en un entorno real, basando dichos modelos en
    los resultados y pruebas llevadas a cabo previamente.

\end{itemize}

%%%%%%%%%%%%%%%%%%%%%%%%%%%%%%%%%%%%%%%%%%%%%%%%%%%%%%%%%%%%%%%%%%%%%%%%%%%%%%%%
%%%%%%%%%%%%%%%%%%%%%%%%%%%%%%%%%%%%%%%%%%%%%%%%%%%%%%%%%%%%%%%%%%%%%%%%%%%%%%%%

\section{Estructura del Documento} \label{sct:intro_estructura}

Esta memoria, que cubre el trabajo realizado en este TFG, se divide en los 
diferentes apartados, los cuales cubren los objetivos establecidos en la 
\textit{\autoref{sct:intro_objetivos} Objetivos}.

Tras esta introducción, en el \textit{\autoref{chp:state-of-the-art} Trabajo relacionado y Estado 
del Arte}, se presenta el trabajo externo que se ha llevado a cabo previamente 
en este proyecto, así como la situación actual de los sistemas involucrados en el 
desarrollo e implementación del mismo, los problemas actuales de estos sistemas, y
diferenciando entre los distintos tipos de sistemas de auto-escalado.

Una vez presentada la situación en la que se ha desarrollado este proyecto, en 
el \textit{\autoref{chp:desarrollo} Desarrollo de un generador de cargas de trabajo para 
sistemas publicador/subscriptor basados en contenido}, se exponen las 
decisiones de diseño e implementación de dicho generador, los sistemas y 
herramientas que se han usado, y análisis de los resultados obtenidos de aplicar 
este generador a E-SilboPS, el sistema usado.

Con el objetivo de mejorar las predicciones del sistema, se presenta, en el
\textit{\autoref{chp:modelos} Generación de  modelos predictivos para el auto-escalado
de E-SilboPS}, el diseño  y la implementación de los modelos predictivos para
E-SilboPS, exponiendo el objetivo de estos modelos, su base y los resultados 
esperados de la aplicación de estos modelos.

A continuación, en el \textit{\autoref{chp:impacto} Impacto del trabajo}, se relacionan
las consecuencias y el impacto general de este proyecto, así como su relación 
con los Objetivos de Desarrollo Sostenible, y en cuáles se enmarca el mismo.

Por último, se presentan, en el \textit{\autoref{chp:resultados} Resultados y conclusiones},
la información obtenida gracias a este proyecto y los resultados que ha generado. 
De igual manera, se explican las conclusiones personales del autor, así como el
trabajo futuro del proyecto, enfocado a futuras mejoras del auto-escalado, la
aplicación de los modelos predictivos, y mejoras en el sistema.

%%%%%%%%%%%%%%%%%%%%%%%%%%%%%%%%%%%%%%%%%%%%%%%%%%%%%%%%%%%%%%%%%%%%%%%%%%%%%%%%
%%%%%%%%%%%%%%%%%%%%%%%%%%%%%%%%%%%%%%%%%%%%%%%%%%%%%%%%%%%%%%%%%%%%%%%%%%%%%%%%