%%***********************************************
%% Plantilla para TFG.
%% Escuela Técnica Superior de Ingenieros Informáticos. UPM.
%%***********************************************

%%-----------------------------------------------
%% Importar Preámbulo:
% -*-coding: utf-8 -*-
%%***********************************************
%% Plantilla para TFG.
%% Escuela Técnica Superior de Ingenieros Informáticos. UPM.
%%***********************************************
%% Preámbulo del documento.
%%***********************************************
\documentclass[a4paper,11pt,openany]{book}
\usepackage[utf8]{inputenc}
\usepackage[T1]{fontenc}
\usepackage[english,spanish,es-lcroman]{babel}
\usepackage{bookman}
\decimalpoint
\usepackage{graphicx}
\usepackage{amsfonts,amsgen,amsmath,amssymb}
\usepackage[top=3cm, bottom=3cm, right=2.54cm, left=2.54cm]{geometry}
\usepackage{afterpage}
\usepackage{colortbl,longtable}
\usepackage[
  pdftex,
  pdfauthor={Miguel Alonso, Carlos},
  pdftitle={Trabajo de Fin de Grado},
  pdfborder={0 0 0}
]{hyperref}
\usepackage{pdfpages}
\usepackage{url}
\usepackage[stable]{footmisc}
\usepackage{parskip} % para separar párrafos con espacio.

\usepackage{lscape}
\usepackage{pgfgantt}

% --- BIBLIOGRAPHY ---
\usepackage[
  backend=biber,
  style=numeric,
  sorting=none
]{biblatex}
\bibliography{bibliography}
\usepackage{csquotes}
% --- BIBLIOGRAPHY ---

\usepackage{float}
\usepackage{subfigure}

%%-----------------------------------------------
\usepackage{fancyhdr}
\pagestyle{fancy}
\fancyhf{}
\fancyhead[LO]{\leftmark}
\fancyhead[RE]{\rightmark}
\setlength{\headheight}{1.5\headheight}
\cfoot{\thepage}

\addto\captionsspanish{ \renewcommand{\contentsname}
  {Índice general} }
\setcounter{tocdepth}{4}
\setcounter{secnumdepth}{4}

\renewcommand{\chaptermark}[1]{\markboth{\textbf{#1}}{}}
\renewcommand{\sectionmark}[1]{\markright{\textbf{\thesection. #1}}}
\newcommand{\HRule}{\rule{\linewidth}{0.5mm}}
\newcommand{\bigrule}{\titlerule[0.5mm]}

\usepackage{appendix}
\renewcommand{\appendixname}{Anexos}
\renewcommand{\appendixtocname}{Anexos}
\renewcommand{\appendixpagename}{}
%%-----------------------------------------------
%% Páginas en blanco sin cabecera:
%%-----------------------------------------------
\usepackage{dcolumn}
\newcolumntype{.}{D{.}{\esperiod}{-1}}
\makeatletter
\addto\shorthandsspanish{\let\esperiod\es@period@code}

\def\clearpage{
  \ifvmode
  \ifnum \@dbltopnum =\m@ne
  \ifdim \pagetotal <\topskip
  \hbox{}
  \fi
  \fi
  \fi
  \newpage
  \thispagestyle{empty}
  \write\m@ne{}
  \vbox{}
  \penalty -\@Mi
}
\makeatother
%%-----------------------------------------------
%% Estilos código de lenguajes: Consola, C, C++ y Python
%%-----------------------------------------------
\usepackage{color}

\definecolor{gray97}{gray}{.97}
\definecolor{gray75}{gray}{.75}
\definecolor{gray45}{gray}{.45}

\usepackage{listings}
\lstset{ frame=Ltb,
  framerule=0pt,
  aboveskip=0.5cm,
  framextopmargin=3pt,
  framexbottommargin=3pt,
  framexleftmargin=0.4cm,
  framesep=0pt,
  rulesep=.0pt,
  backgroundcolor=\color{gray97},
  rulesepcolor=\color{black},
  %
  stringstyle=\ttfamily,
  showstringspaces = false,
  basicstyle=\scriptsize\ttfamily,
  commentstyle=\color{gray45},
  keywordstyle=\bfseries,
  %
  numbers=left,
  numbersep=6pt,
  numberstyle=\tiny,
  numberfirstline = false,
  breaklines=true,
}
\lstnewenvironment{listing}[1][]
                  {\lstset{#1}\pagebreak[0]}{\pagebreak[0]}
                    \lstdefinestyle{consola}{
                                    basicstyle=\scriptsize\bf\ttfamily,
                                    backgroundcolor=\color{gray97}}
                    \lstdefinestyle{C}{
                                    basicstyle=\scriptsize,
                                    frame=single,
                                    language=C,
                                    numbers=left}
                    \lstdefinestyle{C++}{
                                    basicstyle=\small,
                                    frame=single,
                                    backgroundcolor=\color{gray75},
                                    language=C++,
                                    numbers=left}
                    \lstdefinestyle{Java}{
                                    basicstyle=\small,
                                    frame=single,
                                    backgroundcolor=\color{gray75},
                                    language=Java,
                                    numbers=left}
                                \makeatother


%%-----------------------------------------------
%% Cargar datos relativos al TFG:
%% (actualizar estos datos en secciones/_DatosTFG.tex)
%***********************************************
%% Plantilla para TFG.
%% Escuela Técnica Superior de Ingenieros Informáticos. UPM.
%%***********************************************
%% Información requerida para completar la portada.
%%*********************************************** 

%% Escribe Nombre y Apellidos del autor del trabajo:
\newcommand{\NombreAutor}{ Carlos Miguel Alonso }

%% Escribe el Grado: 
\newcommand{\Grado}{ Ingeniería Informática }

%% Escribe el Título del Trabajo:
\newcommand{\TituloTFG}{ Mejora de un Sistema de Auto-escalado para Sistemas Distribuidos } 

%% Escribe Nombre y Apellidos del Tutor del trabajo: 
\newcommand{\NombreTutor}{ Victor Rampérez Martín } 

% Escribe el Departamento al que pertenece el Tutor:
\newcommand{\Departamento}{ Departamento de Lenguajes, Sistemas Informáticos e Ingeniería del Software }

% Escribe la fecha de lectura, en formato: Mes - Año
\newcommand{\Fecha}{ Febrero - 2022 }
%%***********************************************


%%-----------------------------------------------
%% Documento
\begin{document}
\input{include/Portada}

%%-----------------------------------------------
\chapter*{Descripción y Objetivos}
En este Trabajo de Fin de Grado se desarrollará y expondrá el trabajo y los avances realizados en el proyecto de mejora de un sistema de auto-escalado para sistemas distribuidos de forma que el uso de recursos de estos sistemas sea el más eficiente posible, aumentando y disminuyendo los recursos disponibles en base a los resultados de las predicciones de de la carga de trabajo en el futuro inmediato.

Para llevar a cabo dicho objetivo, lo primero será comprender el funcionamiento de los sistemas de auto-escalado por medio de la lectura de artículos científicos, y documentación del sistema que se va a usar, al igual que conocer el estado actual de los sistemas de auto-escalado, sus ventajas y limitaciones. Para poder simular una situación real en el sistema, se necesita una carga de trabajo real, las cuales no son fáciles de encontrar, principalmente por problemas de seguridad y confidencialidad de los datos contenidos en dichas cargas.

Una vez obtenida esta carga, se desarrollarán pruebas de rendimiento de forma que el sistema experimente una situación real, mientras que se guardan las medidas de rendimiento más relevantes para mostrar el estado del sistema en cada momento, pudiendo generar gráficas y diagramas de cómo ha reaccionado el sistema. Con estos resultados, se pueden crear diferentes cargas de trabajo basadas en la real y usando las medidas de rendimiento obtenidas, pudiendo desarrollar más pruebas que lleven el sistema a ciertas situaciones, por ejemplo, que el sistema sature.

Tras probar el sistema con la cargas de trabajo de forma extensiva, se aplicarán modelos predictivos a las medidas de rendimiento más relevantes del sistema, de forma que este se pueda anticipar a posibles situaciones especiales (e.g. incremento repentino de la carga de trabajo) y actuar en consecuencia.

Los principales objetivos de este Trabajo de Fin de Grado son:
\begin{itemize}
    \item[•] Comprender el funcionamiento de los sistemas de auto-escalado
    \item[•] Encontrar una carga de trabajo real, compatible y exportable al sistema usado
    \item[•] Desarrollar pruebas para que el sistema de auto-escalado utilice esta carga
    \item[•] Interpretar las medidas de rendimiento resultantes de las pruebas
    \item[•] Desarrollar nuevas cargas de trabajo basadas en la carga real
    \item[•] Desarrollar más pruebas que utilicen estas nuevas cargas de trabajo
    \item[•] Interpretar las nuevas medidas de rendimiento resultantes de las pruebas
    \item[•] Implementar modelos predictivos en base a los resultados obtenidos
\end{itemize}


\chapter*{Tareas}
Partiendo de los objetivos establecidos en este Trabajo de Fin de Grado, se obtiene la siguiente lista de tareas a realizar para cumplir dichos objetivos, la cual se usará como referencia para la planificación de estas tareas a lo largo del tiempo, en el diagrama de Gantt de la siguiente página:\\

\begin{enumerate}
    \item[T1 -] Lectura de documentación sobre sistemas de auto-escalado
    \item[T2 -] Familiarización con el sistema de auto-escalado a usar
    \item[T3 -] Buscar una carga de trabajo real y compatible
    \item[T4 -] Comprobar la compatibilidad e integridad de la carga de trabajo
    \item[T5 -] Desarrollar un traductor (parser) que convierta la carga de trabajo a formato compatible con el sistema usado
    \item[T6 -] Comprobar el correcto funcionamiento del traductor desarrollado
    \item[T7 -] Desarrollar pruebas en el sistema que utilicen la carga de trabajo
    \item[T8 -] Desarrollar scripts que muestren e interpreten las medidas de rendimiento
    \item[T9 -] Crear nuevas cargas de trabajo basadas en la real y apoyadas en los resultados obtenidos
    \item[T10 -] Desarrollar nuevas pruebas en el sistema con las nuevas cargas de trabajo
    \item[T11 -] Interpretar las nuevas medidas de rendimiento con los scripts ya creados
    \item[T12 -] Documentar condiciones que llevan al sistema a situaciones destacables
    \item[T13 -] Aplicar modelos predictivos a las medidas de rendimiento del sistema
\end{enumerate}


\begin{landscape}
% \ganttbar{NAME}{START}{END} \\
% \ganttlink[link type={s-s || s-f || f-s || f-f}]{elem0}{elem1}
\begin{center}
    \begin{ganttchart}[
            hgrid,
            vgrid,
            x unit=1cm,2
            y unit chart=0.9cm
        ]{1}{17}
        \gantttitle{Semana}{17} \\
        \gantttitlelist{1,...,17}{1} \\
        \ganttbar[name=T1]{T1}{1}{2} \\
        \ganttbar[name=T2]{T2}{3}{3} \\
        \ganttbar[name=T3]{T3}{4}{5} \\
        \ganttbar[name=T4]{T4}{6}{6} \\
        \ganttbar[name=T5]{T5}{6}{7} \\
        \ganttbar[name=T6]{T6}{7}{7} \\
        \ganttbar[name=T7]{T7}{8}{10} \\
        \ganttbar[name=T8]{T8}{11}{12} \\
        \ganttbar[name=T9]{T9}{13}{14} \\
        \ganttbar[name=T10]{T10}{14}{14} \\
        \ganttbar[name=T11]{T11}{15}{15} \\
        \ganttbar[name=T12]{T12}{16}{16} \\
        \ganttbar[name=T13]{T13}{16}{17} \\
        \ganttbar[name=TFG]{Doc. TFG}{2}{15} \\
        \ganttbar[name=Pres]{Pres. Defensa}{16}{17}
        % Links between tasks/bars
        \ganttlink[link type=f-s]{T1}{T2}
        \ganttlink[link type=f-s]{T2}{T3}
        \ganttlink[link type=f-s]{T3}{T4}
        \ganttlink[link type=s-s]{T4}{T5}
        \ganttlink[link type=f-f]{T5}{T6}
        \ganttlink[link type=f-s]{T6}{T7}
        \ganttlink[link type=f-s]{T7}{T8}
        \ganttlink[link type=f-s]{T8}{T9}
        \ganttlink[link type=f-f]{T9}{T10}
        \ganttlink[link type=f-s]{T10}{T11}
        \ganttlink[link type=f-s]{T11}{T12}
        \ganttlink[link type=s-s]{T12}{T13}
    \end{ganttchart}
\end{center}

\end{landscape}


\chapter*{Propuesta de Trabajo Oficial}

%%%%%%%%%%%%%%%%%%%%%%%%%%%%%%%%%%%%%%%%%%%%%%%%%%%%%%%%%%%
%% Final del plan de trabajo
%%%%%%%%%%%%%%%%%%%%%%%%%%%%%%%%%%%%%%%%%%%%%%%%%%%%%%%%%%%

%%-----------------------------------------------
{}\end{document}
