%%***********************************************
%% Plantilla para TFG.
%% Escuela Técnica Superior de Ingenieros Informáticos. UPM.
%%***********************************************

%%-----------------------------------------------
%% Importar Preámbulo:
% -*-coding: utf-8 -*-
%%***********************************************
%% Plantilla para TFG.
%% Escuela Técnica Superior de Ingenieros Informáticos. UPM.
%%***********************************************
%% Preámbulo del documento.
%%***********************************************
\documentclass[a4paper,11pt,openany]{book}
\usepackage[utf8]{inputenc}
\usepackage[T1]{fontenc}
\usepackage[english,spanish,es-lcroman]{babel}
\usepackage{bookman}
\decimalpoint
\usepackage{graphicx}
\usepackage{amsfonts,amsgen,amsmath,amssymb}
\usepackage[top=3cm, bottom=3cm, right=2.54cm, left=2.54cm]{geometry}
\usepackage{afterpage}
\usepackage{colortbl,longtable}
\usepackage[
  pdftex,
  pdfauthor={Miguel Alonso, Carlos},
  pdftitle={Trabajo de Fin de Grado},
  pdfborder={0 0 0}
]{hyperref}
\usepackage{pdfpages}
\usepackage{url}
\usepackage[stable]{footmisc}
\usepackage{parskip} % para separar párrafos con espacio.

\usepackage{lscape}
\usepackage{pgfgantt}

% --- BIBLIOGRAPHY ---
\usepackage[
  backend=biber,
  style=numeric,
  sorting=none
]{biblatex}
\bibliography{bibliography}
\usepackage{csquotes}
% --- BIBLIOGRAPHY ---

\usepackage{float}
\usepackage{subfigure}

%%-----------------------------------------------
\usepackage{fancyhdr}
\pagestyle{fancy}
\fancyhf{}
\fancyhead[LO]{\leftmark}
\fancyhead[RE]{\rightmark}
\setlength{\headheight}{1.5\headheight}
\cfoot{\thepage}

\addto\captionsspanish{ \renewcommand{\contentsname}
  {Índice general} }
\setcounter{tocdepth}{4}
\setcounter{secnumdepth}{4}

\renewcommand{\chaptermark}[1]{\markboth{\textbf{#1}}{}}
\renewcommand{\sectionmark}[1]{\markright{\textbf{\thesection. #1}}}
\newcommand{\HRule}{\rule{\linewidth}{0.5mm}}
\newcommand{\bigrule}{\titlerule[0.5mm]}

\usepackage{appendix}
\renewcommand{\appendixname}{Anexos}
\renewcommand{\appendixtocname}{Anexos}
\renewcommand{\appendixpagename}{}
%%-----------------------------------------------
%% Páginas en blanco sin cabecera:
%%-----------------------------------------------
\usepackage{dcolumn}
\newcolumntype{.}{D{.}{\esperiod}{-1}}
\makeatletter
\addto\shorthandsspanish{\let\esperiod\es@period@code}

\def\clearpage{
  \ifvmode
  \ifnum \@dbltopnum =\m@ne
  \ifdim \pagetotal <\topskip
  \hbox{}
  \fi
  \fi
  \fi
  \newpage
  \thispagestyle{empty}
  \write\m@ne{}
  \vbox{}
  \penalty -\@Mi
}
\makeatother
%%-----------------------------------------------
%% Estilos código de lenguajes: Consola, C, C++ y Python
%%-----------------------------------------------
\usepackage{color}

\definecolor{gray97}{gray}{.97}
\definecolor{gray75}{gray}{.75}
\definecolor{gray45}{gray}{.45}

\usepackage{listings}
\lstset{ frame=Ltb,
  framerule=0pt,
  aboveskip=0.5cm,
  framextopmargin=3pt,
  framexbottommargin=3pt,
  framexleftmargin=0.4cm,
  framesep=0pt,
  rulesep=.0pt,
  backgroundcolor=\color{gray97},
  rulesepcolor=\color{black},
  %
  stringstyle=\ttfamily,
  showstringspaces = false,
  basicstyle=\scriptsize\ttfamily,
  commentstyle=\color{gray45},
  keywordstyle=\bfseries,
  %
  numbers=left,
  numbersep=6pt,
  numberstyle=\tiny,
  numberfirstline = false,
  breaklines=true,
}
\lstnewenvironment{listing}[1][]
                  {\lstset{#1}\pagebreak[0]}{\pagebreak[0]}
                    \lstdefinestyle{consola}{
                                    basicstyle=\scriptsize\bf\ttfamily,
                                    backgroundcolor=\color{gray97}}
                    \lstdefinestyle{C}{
                                    basicstyle=\scriptsize,
                                    frame=single,
                                    language=C,
                                    numbers=left}
                    \lstdefinestyle{C++}{
                                    basicstyle=\small,
                                    frame=single,
                                    backgroundcolor=\color{gray75},
                                    language=C++,
                                    numbers=left}
                    \lstdefinestyle{Java}{
                                    basicstyle=\small,
                                    frame=single,
                                    backgroundcolor=\color{gray75},
                                    language=Java,
                                    numbers=left}
                                \makeatother


%%-----------------------------------------------
%% Cargar datos relativos al TFG:
%% (actualizar estos datos en secciones/_DatosTFG.tex)
%***********************************************
%% Plantilla para TFG.
%% Escuela Técnica Superior de Ingenieros Informáticos. UPM.
%%***********************************************
%% Información requerida para completar la portada.
%%*********************************************** 

%% Escribe Nombre y Apellidos del autor del trabajo:
\newcommand{\NombreAutor}{ Carlos Miguel Alonso }

%% Escribe el Grado: 
\newcommand{\Grado}{ Ingeniería Informática }

%% Escribe el Título del Trabajo:
\newcommand{\TituloTFG}{ Mejora de un Sistema de Auto-escalado para Sistemas Distribuidos } 

%% Escribe Nombre y Apellidos del Tutor del trabajo: 
\newcommand{\NombreTutor}{ Victor Rampérez Martín } 

% Escribe el Departamento al que pertenece el Tutor:
\newcommand{\Departamento}{ Departamento de Lenguajes, Sistemas Informáticos e Ingeniería del Software }

% Escribe la fecha de lectura, en formato: Mes - Año
\newcommand{\Fecha}{ Febrero - 2022 }
%%***********************************************


%%-----------------------------------------------
%% Documento
\begin{document}
\input{include/Portada}

%%-----------------------------------------------

%%%%%%%%%%%%%%%%%%%%%%%%%%%%%%%%%%%%%%%%%%%%%%%%%%%%%%%%%%%%%%%%%%%%%%%%%%%%%%%%
%%%%%%%%%%%%%%%%%%%%%%%%%%%%%%%%%%%%%%%%%%%%%%%%%%%%%%%%%%%%%%%%%%%%%%%%%%%%%%%%

\chapter*{Resumen del trabajo realizado}

Durante estos meses, se ha comenzado con la lectura de documentación, para poder comprender
el funcionamiento de los sistemas que se han usado y poder llevar a cabo las tareas y cumplir
con los objetivos establecidos.

Lo siguiente que se ha hecho es buscar una carga de trabajo real que se adapte a los requisitos
establecidos para poder ser utilizado en las tareas que se tienen planificadas. Esta carga
se ha encontrado en el \textit{Middleware Systems Research Group},
de la de la Universidad de Toronto. Esta carga, tras ser analizada, se ha confirmado su utilidad,
y se ha usado para continuar con el desarrollo e implementar las pruebas siguientes del proyecto.

Una vez encontrada la carga de trabajo a usar, se ha procedido con el diseño y la implementación del
generador de cargas de trabajo, que utiliza esta carga para crear una análoga pero con el formato
del sistema utilizado.

Tras implementar este generador y obtener la carga compatible, se ha procedido a desarrollar diferentes
pruebas, obteniendo de cada una de ellas las métricas de rendimiento, creando a su vez pequeños
programas para poder visualizar estas métricas mediante gráficos.

El trabajo actual consiste en la interpretación de los datos de rendimiento obtenidos, para poder
implementar pruebas que usen la carga y que lleven al sistema a situaciones en las que queremos ver
de forma específica su rendimiento.

%%%%%%%%%%%%%%%%%%%%%%%%%%%%%%%%%%%%%%%%%%%%%%%%%%%%%%%%%%%%%%%%%%%%%%%%%%%%%%%%
%%%%%%%%%%%%%%%%%%%%%%%%%%%%%%%%%%%%%%%%%%%%%%%%%%%%%%%%%%%%%%%%%%%%%%%%%%%%%%%%

\chapter*{Modificaciones del Plan de Trabajo}

Ni los objetivos ni las tareas de este proyecto se han modificado, manteniendo el objetivo último del mismo.

El Diagrama de Gantt tampoco ha sufrido modificaciones, a pesar de que han surgido problemas con el diseño
e implementación de ciertas pruebas, lo que han ralentizado el avance del proyecto, pero no ha requerido
de modificar el plan de trabajo.

\section*{Objetivos}

\begin{itemize}
    \item[•] Comprender el funcionamiento de los sistemas de auto-escalado
    \item[•] Encontrar una carga de trabajo real, compatible y exportable al sistema usado
    \item[•] Desarrollar pruebas para que el sistema de auto-escalado utilice esta carga
    \item[•] Interpretar las métricas de rendimiento resultantes de las pruebas
    \item[•] Desarrollar nuevas cargas de trabajo basadas en la carga real
    \item[•] Desarrollar más pruebas que utilicen estas nuevas cargas de trabajo
    \item[•] Interpretar las nuevas métricas de rendimiento resultantes de las pruebas
    \item[•] Implementar modelos predictivos en base a los resultados obtenidos
\end{itemize}

%%%%%%%%%%%%%%%%%%%%%%%%%%%%%%%%%%%%%%%%%%%%%%%%%%%%%%%%%%%%%%%%%%%%%%%%%%%%%%%%
%%%%%%%%%%%%%%%%%%%%%%%%%%%%%%%%%%%%%%%%%%%%%%%%%%%%%%%%%%%%%%%%%%%%%%%%%%%%%%%%

\section*{Tareas}

\begin{enumerate}
    \item[T1 -] Lectura de documentación sobre sistemas de auto-escalado
    \item[T2 -] Familiarización con el sistema de auto-escalado a usar
    \item[T3 -] Buscar una carga de trabajo real y compatible
    \item[T4 -] Comprobar la compatibilidad e integridad de la carga de trabajo
    \item[T5 -] Desarrollar un generador que convierta la carga de trabajo a formato compatible con el sistema usado
    \item[T6 -] Comprobar el correcto funcionamiento del generador desarrollado
    \item[T7 -] Desarrollar pruebas en el sistema que utilicen la carga de trabajo
    \item[T8 -] Desarrollar scripts que muestren e interpreten las medidas de rendimiento
    \item[T9 -] Crear nuevas cargas de trabajo basadas en la real y apoyadas en los resultados obtenidos
    \item[T10 -] Desarrollar nuevas pruebas en el sistema con las nuevas cargas de trabajo
    \item[T11 -] Interpretar las nuevas medidas de rendimiento con los scripts ya creados
    \item[T12 -] Documentar condiciones que llevan al sistema a situaciones destacables
    \item[T13 -] Aplicar modelos predictivos a las medidas de rendimiento del sistema
\end{enumerate}


\begin{landscape}
\input{include/GanttDiagram_revision}
\end{landscape}

%%%%%%%%%%%%%%%%%%%%%%%%%%%%%%%%%%%%%%%%%%%%%%%%%%%%%%%%%%%
%% Final del plan de trabajo
%%%%%%%%%%%%%%%%%%%%%%%%%%%%%%%%%%%%%%%%%%%%%%%%%%%%%%%%%%%

%%-----------------------------------------------
{}\end{document}
